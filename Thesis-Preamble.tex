%%%%%%%%%%%%%%%%%%%%%%%%%%%%%%%%%%%%%%%%%%%%%%%%%%%%%%%%%%%%%%%%%
%% ANOTHER LATEX THESIS TEMPLATE.
%% https://github.com/zachscrivena/another-latex-thesis-template
%% This is free and unencumbered software released into the
%% public domain; see <http://unlicense.org> for details.
%%%%%%%%%%%%%%%%%%%%%%%%%%%%%%%%%%%%%%%%%%%%%%%%%%%%%%%%%%%%%%%%%

%%%%%%%%%%%%%%%%%%%%%%%%%%%%%%%%%%%%%%%%%%%%%%%%%%%%%%%%%%%%%%%%%
%% LATEX COMPILATION SETTINGS.
%%%%%%%%%%%%%%%%%%%%%%%%%%%%%%%%%%%%%%%%%%%%%%%%%%%%%%%%%%%%%%%%%

% Run LaTeX in non-stop mode.
%\ifnum\TypesetInNonStopMode=1
%\nonstopmode
%\fi

%%%%%%%%%%%%%%%%%%%%%%%%%%%%%%%%%%%%%%%%%%%%%%%%%%%%%%%%%%%%%%%%%
%% PAGE SIZE AND MARGINS.
%%%%%%%%%%%%%%%%%%%%%%%%%%%%%%%%%%%%%%%%%%%%%%%%%%%%%%%%%%%%%%%%%

% Letter-size (8.5in x 11in) single-sided pages,
% with margins (top,right,bottom,left) = (1,1,1,1.5)in,
% and header 0.25in above the text body.
% \usepackage[
% total={6in,9in},
% top=1in,
% left=1.5in,
% headsep=0.25in]{geometry}
\usepackage[paper=A4,oneside,DIV=calc,BCOR=5mm]{typearea}
\usepackage[fontsize=13pt]{scrextend}

%%%%%%%%%%%%%%%%%%%%%%%%%%%%%%%%%%%%%%%%%%%%%%%%%%%%%%%%%%%%%%%%%
%% FONTS.
%%%%%%%%%%%%%%%%%%%%%%%%%%%%%%%%%%%%%%%%%%%%%%%%%%%%%%%%%%%%%%%%%

\usepackage{ifxetex,ifluatex}
\usepackage{amssymb,amsmath}
\ifnum 0\ifxetex 1\fi\ifluatex 1\fi=0 % if pdftex
\usepackage[T5]{fontenc}
\usepackage[utf8]{inputenc}
%\usepackage{lmodern} % For Latin Modern fonts.
\usepackage{times} % For Times fonts.
%\usepackage{tgtermes}
% Sans-serif fonts.
\usepackage[scaled=0.89]{helvet}
\renewcommand{\sffamily}{\usefont{T5}{phv}{m}{n}}
\newcommand{\UseLMSSBoldFont}{\usefont{T5}{lmss}{m}{n}\bfseries}

% Monospace (typewriter) font.
\usepackage[scaled=0.78]{beramono}
\renewcommand{\ttfamily}{\usefont{T5}{fvm}{m}{n}}

\else % if luatex or xelatex
\ifxetex
\usepackage{mathspec}
\else
\usepackage{fontspec}
\fi
\usepackage{unicode-math}
\defaultfontfeatures{Ligatures=TeX,Scale=MatchLowercase}
%\setmainfont{TeX Gyre Termes}
\setmainfont{Times New Roman}
\setmonofont{TeX Gyre Cursor}
% \setsansfont{TeX Gyre Heros}
\setsansfont{Arial}
\setmathfont{XITS Math}
\newcommand{\UseLMSSBoldFont}{\sffamily\bfseries}
\fi
\usepackage{microtype}

%%%%%%%%%%%%%%%%%%%%%%%%%%%%%%%%%%%%%%%%%%%%%%%%%%%%%%%%%%%%%%%%%
%% MISCELLANEOUS PACKAGES.
%%%%%%%%%%%%%%%%%%%%%%%%%%%%%%%%%%%%%%%%%%%%%%%%%%%%%%%%%%%%%%%%%

\usepackage[vietnamese]{babel} % For language-specific hyphenation.
\usepackage{cite} % Automatically sort and range citations numbers.
\usepackage{environ} % For easy definition of environments.
\usepackage{rotating} % For rotating objects.
\usepackage{framed} % For framed text.

%%%%%%%%%%%%%%%%%%%%%%%%%%%%%%%%%%%%%%%%%%%%%%%%%%%%%%%%%%%%%%%%%
%% PDF OUTPUT.
%%%%%%%%%%%%%%%%%%%%%%%%%%%%%%%%%%%%%%%%%%%%%%%%%%%%%%%%%%%%%%%%%

% PDF settings and properties (set options with \hypersetup{}).
\usepackage{hyperref} % For TEX --> DVI --> PDF.
% \usepackage[pdftex]{hyperref} % For TEX --> PDF.
\usepackage{bookmark}

%%%%%%%%%%%%%%%%%%%%%%%%%%%%%%%%%%%%%%%%%%%%%%%%%%%%%%%%%%%%%%%%%
%% SECTION HEADINGS.
%%%%%%%%%%%%%%%%%%%%%%%%%%%%%%%%%%%%%%%%%%%%%%%%%%%%%%%%%%%%%%%%%

% Section heading fonts.
\usepackage{sectsty} % For selecting fonts for section headings.
\allsectionsfont{\UseLMSSBoldFont}

% Section numbering depth.
\setcounter{secnumdepth}{10}

%%%%%%%%%%%%%%%%%%%%%%%%%%%%%%%%%%%%%%%%%%%%%%%%%%%%%%%%%%%%%%%%%
%% PARAGRAPHS.
%%%%%%%%%%%%%%%%%%%%%%%%%%%%%%%%%%%%%%%%%%%%%%%%%%%%%%%%%%%%%%%%%

% Line spacing.
\usepackage{setspace}
%\singlespacing
\onehalfspacing
%\doublespacing
%\setstretch{1.6} % custom

% Paragraph indentation:
% Indent first line of all paragraphs (including the first),
% as in IEEE style.
\makeatletter
\let\@afterindentfalse\@afterindenttrue
\makeatother

% Indented blocks.
\newcommand{\IndentBlock}[1]{\noindent\hangafter=0\hangindent=#1\parindent\ignorespaces}
\newcommand{\IndentHanging}{\noindent\hangafter=1\hangindent=\parindent\ignorespaces}

%%%%%%%%%%%%%%%%%%%%%%%%%%%%%%%%%%%%%%%%%%%%%%%%%%%%%%%%%%%%%%%%%
%% HEADERS AND FOOTERS.
%%%%%%%%%%%%%%%%%%%%%%%%%%%%%%%%%%%%%%%%%%%%%%%%%%%%%%%%%%%%%%%%%

% Header.
\newcommand{\HeaderText}{\hfill\footnotesize\sffamily\thepage\hfill}

% Footer.
\ifnum\TypesetInDraftMode=1
\newcommand{\FooterText}{\hfill\footnotesize\sffamily\color{red}{DRAFT}~\Timestamp\hfill}
\fi

\ifnum\TypesetInDraftMode=0
\newcommand{\FooterText}{}
\fi

\makeatletter
\def\ps@plain{%
\def\@oddhead{\HeaderText}%
\def\@evenhead{\HeaderText}%
\def\@oddfoot{\FooterText}%
\def\@evenfoot{\FooterText}}

\makeatletter
\def\ps@empty{%
\def\@oddhead{}%
\def\@evenhead{}%
\def\@oddfoot{\FooterText}%
\def\@evenfoot{\FooterText}}

\pagestyle{plain}

%%%%%%%%%%%%%%%%%%%%%%%%%%%%%%%%%%%%%%%%%%%%%%%%%%%%%%%%%%%%%%%%%
%% FOOTNOTES.
%%%%%%%%%%%%%%%%%%%%%%%%%%%%%%%%%%%%%%%%%%%%%%%%%%%%%%%%%%%%%%%%%

% Blank footnotes.
\newcommand\BlankFootnote[1]{%
\begingroup%
\renewcommand{\thefootnote}{}%
\footnotetext{#1}%
\addtocounter{footnote}{-1}%
\addtocounter{Hfootnote}{-1}%
\endgroup}

%%%%%%%%%%%%%%%%%%%%%%%%%%%%%%%%%%%%%%%%%%%%%%%%%%%%%%%%%%%%%%%%%
%% LISTS.
%%%%%%%%%%%%%%%%%%%%%%%%%%%%%%%%%%%%%%%%%%%%%%%%%%%%%%%%%%%%%%%%%

% Numbered lists in IEEE style.
% (Individual lists can be modified by redefining
% these macros inside the enumerate environment.)
\makeatletter
% 1st level: 1), 2), 3)
\renewcommand{\theenumi}{\arabic{enumi}}
\renewcommand{\labelenumi}{\theenumi)}
% 2nd level: a), b), c)
\renewcommand{\theenumii}{\alph{enumii}}
\renewcommand{\labelenumii}{\theenumii)}
\renewcommand\p@enumii{}
% 3rd level: i), ii), iii)
\renewcommand{\theenumiii}{\roman{enumiii}}
\renewcommand{\labelenumiii}{\theenumiii)}
\renewcommand\p@enumiii{}
% 4th level: A), B), C)
\renewcommand{\theenumiv}{\Alph{enumiv}}
\renewcommand{\labelenumiv}{\theenumiv)}
\renewcommand\p@enumiv{}
\makeatother

% Definition items.
\newcommand{\DefineItem}[1]{%
\IndentBlock{1}#1\nopagebreak
\par\IndentBlock{2}}

%%%%%%%%%%%%%%%%%%%%%%%%%%%%%%%%%%%%%%%%%%%%%%%%%%%%%%%%%%%%%%%%%
%% FIGURES AND TABLES.
%%%%%%%%%%%%%%%%%%%%%%%%%%%%%%%%%%%%%%%%%%%%%%%%%%%%%%%%%%%%%%%%%

\usepackage{graphicx} % To support graphics in EPS format.
\usepackage{multirow} % To support multi-row cells in tables.
\usepackage{booktabs} % For making nice tables.

% Adjust spacing between table rows.
\renewcommand*\arraystretch{1.25}

% Dashed lines in tables.
\usepackage{arydshln}
\def\dashvertical{;{2pt/3pt}}
\def\dashhorizontal{\hdashline[2pt/3pt]}

% Captions for figures and tables.
\newcommand{\CaptionFontSize}{\small}

\makeatletter
\def\@figurestring{figure}
\def\@tablestring{table}
\def\@makecaption#1#2{%
\CaptionFontSize
\ifx\@captype\@figurestring
\vskip1em
\fi
\sbox\@tempboxa{{\sffamily\bfseries{#1.}}\hspace{0.5em}#2}%
\ifdim\wd\@tempboxa>\hsize
{{\sffamily\bfseries{#1.}}\hspace{0.5em}#2}%
\else
\hb@xt@\hsize{\hfil\box\@tempboxa\hfil}%
\fi
\ifx\@captype\@tablestring
\vskip1em
\fi
}
\makeatother

%%%%%%%%%%%%%%%%%%%%%%%%%%%%%%%%%%%%%%%%%%%%%%%%%%%%%%%%%%%%%%%%%
%% COLORS.
%%%%%%%%%%%%%%%%%%%%%%%%%%%%%%%%%%%%%%%%%%%%%%%%%%%%%%%%%%%%%%%%%

\usepackage[usenames]{color} % For colors.
%% \definecolor{MyDarkBlue}{rgb}{0,0.08,0.45}
%% {\color{MyDarkBlue}This text is dark blue}

%%%%%%%%%%%%%%%%%%%%%%%%%%%%%%%%%%%%%%%%%%%%%%%%%%%%%%%%%%%%%%%%%
%% DATE AND TIME.
%%%%%%%%%%%%%%%%%%%%%%%%%%%%%%%%%%%%%%%%%%%%%%%%%%%%%%%%%%%%%%%%%

\usepackage{datetime} % For dates and times.
\renewcommand{\dateseparator}{-}
\settimeformat{xxivtime}

% Timestamp.
\newcommand{\Timestamp}{{\yyyymmdddate\today}~{\currenttime}}

%%%%%%%%%%%%%%%%%%%%%%%%%%%%%%%%%%%%%%%%%%%%%%%%%%%%%%%%%%%%%%%%%
%% COMMENTS, MARKERS, HIDDEN TEXT.
%%%%%%%%%%%%%%%%%%%%%%%%%%%%%%%%%%%%%%%%%%%%%%%%%%%%%%%%%%%%%%%%%

% Comments, markers, hide.
\newcommand{\BoxComment}[1]{{\color{red}\vskip1em\noindent\fbox{\parbox[c]{0.98\columnwidth}{#1}}\vskip1em}}
\newcommand{\Hide}[1]{}

\ifnum\TypesetInDraftMode=1
\newcommand{\TODO}[1]{{\color{red}\fbox{\texttt{TODO}}~#1\color{black}}}
\newcommand{\WIP}{{\BoxComment{~\hfill~\texttt{NOTE: Material below this point is work-in-progress.}~\hfill~}}}
\fi

\ifnum\TypesetInDraftMode=0
\newcommand{\TODO}[1]{}
\newcommand{\WIP}{}
\fi

%%%%%%%%%%%%%%%%%%%%%%%%%%%%%%%%%%%%%%%%%%%%%%%%%%%%%%%%%%%%%%%%%
%% MATHEMATICS.
%%%%%%%%%%%%%%%%%%%%%%%%%%%%%%%%%%%%%%%%%%%%%%%%%%%%%%%%%%%%%%%%%

\usepackage{amsmath,amsfonts,amsbsy,amssymb,amsthm} % AMS packages.

% Indicator function "1[.]" symbol.
% Option 2: Use "bbold" package (install "bbold-type1" first)
\ifnum 0\ifxetex 1\fi\ifluatex 1\fi=0 % if pdftex
\DeclareSymbolFont{bbold}{U}{bbold}{m}{n}
\DeclareSymbolFontAlphabet{\mathbbold}{bbold}
\newcommand{\one}[1]{{\mathbbold{1}\left[#1\right]}}
% Option 2: Use "\mathbf"
\else
\newcommand{\one}[1]{{\mathbf{1}\left[#1\right]}}
\fi
% Option 3: Use "dsfont" package
%\usepackage{dsfont}
%\newcommand{\one}[1]{{\mathds{1}\left[#1\right]}}

% Allow line breaks within math blocks.
\allowdisplaybreaks

% Prevent line breaks within math expressions.
\relpenalty=10000
\binoppenalty=10000
\sloppy

% Theorems (cf. "amsthm.sty").
\newtheoremstyle{MyPlain}%
{0.4em}% space above
{0.4em}% space below
{\itshape}% body font
{}% indent amount
{\sffamily\bfseries}% theorem head font
{.}% punctuation after theorem head
{0.5em}% space after theorem head
{}% theorem head spec

\newtheoremstyle{MyDefinition}%
{0.4em}% space above
{0.4em}% space below
{}% body font
{}% indent amount
{\sffamily\bfseries}% theorem head font
{.}% punctuation after theorem head
{0.5em}% space after theorem head
{}% theorem head spec

\theoremstyle{MyPlain}
\newtheorem{Thm:Theorem}{Theorem}[chapter]
\newtheorem{Thm:Lemma}[Thm:Theorem]{Lemma}
\newtheorem{Thm:Corollary}[Thm:Theorem]{Corollary}
\newtheorem{Thm:Claim}[Thm:Theorem]{Claim}
\newtheorem{Thm:Proposition}[Thm:Theorem]{Proposition}
\newtheorem{Thm:Conjecture}[Thm:Theorem]{Conjecture}

\theoremstyle{MyDefinition}
\newtheorem{Thm:Problem}[Thm:Theorem]{Problem}
\newtheorem{Thm:Definition}[Thm:Theorem]{Definition}
\newtheorem{Thm:Example}[Thm:Theorem]{Example}

\renewenvironment{proof}[1][\proofname]{%
{\par\vskip0.4em\noindent%
\sffamily\bfseries\itshape{#1:}%
\hspace{0.5em}}}%
{\nopagebreak\hspace*{\fill}~\mbox{\rule[0pt]{1.3ex}{1.3ex}}\par}

\newcommand{\qedmarker}{\nopagebreak\hspace*{\fill}~%
\mbox{\rule[0pt]{1.3ex}{1.3ex}}\par}

% Resized "align" environment.
% (may not display correctly in DVI viewer)
\NewEnviron{ResizedAlign}[2]{%
\par\noindent
\resizebox{#1}{!}{
\parbox{#2}{
\begin{align}
\BODY
\end{align}}}\par}

% Resized "align*" environment.
% (may not display correctly in DVI viewer)
\NewEnviron{ResizedAlign*}[2]{%
\par\noindent
\resizebox{#1}{!}{
\parbox{#2}{
\begin{align*}
\BODY
\end{align*}}}\par}

%%%%%%%%%%%%%%%%%%%%%%%%%%%%%%%%%%%%%%%%%%%%%%%%%%%%%%%%%%%%%%%%%
%% CODE.
%%%%%%%%%%%%%%%%%%%%%%%%%%%%%%%%%%%%%%%%%%%%%%%%%%%%%%%%%%%%%%%%%

% Code blocks.
\newcommand{\StartCode}{%
\noindent\ignorespaces%
\begin{framed}
\small\ttfamily
%\def\FrameCommand{{\color{black}{\vrule width 0.5pt\relax\hspace{5pt}}}}%
%\MakeFramed{\advance\hsize-\width\FrameRestore}%
}

\newcommand{\StopCode}{%
\ignorespaces\end{framed}
%\ignorespaces\endMakeFramed%
}

%%%%%%%%%%%%%%%%%%%%%%%%%%%%%%%%%%%%%%%%%%%%%%%%%%%%%%%%%%%%%%%%%
%% TABLE OF CONTENTS (TOC) SETTINGS.
%%%%%%%%%%%%%%%%%%%%%%%%%%%%%%%%%%%%%%%%%%%%%%%%%%%%%%%%%%%%%%%%%

% TOC depth.
\setcounter{tocdepth}{10}

% Suppress entries in the TOC.
\newcommand{\DummyThree}[3]{}

\newcommand{\DisableTOCUpdates}{%
\let\tempaddcontentsline=\addcontentsline
\let\addcontentsline=\DummyThree}

\newcommand{\EnableTOCUpdates}{%
\let\addcontentsline=\tempaddcontentsline}

%%%%%%%%%%%%%%%%%%%%%%%%%%%%%%%%%%%%%%%%%%%%%%%%%%%%%%%%%%%%%%%%%
%% SAMPLE/BLIND TEXT.
%%%%%%%%%%%%%%%%%%%%%%%%%%%%%%%%%%%%%%%%%%%%%%%%%%%%%%%%%%%%%%%%%

\ifnum\TypesetInDraftMode=1
\usepackage{lipsum} % For sample/blind text.
\fi

\ifnum\TypesetInDraftMode=0
\newcommand{\lipsum}[1][]{}
\fi

%%%%%%%%%%%%%%%%%%%%%%%%%%%%%%%%%%%%%%%%%%%%%%%%%%%%%%%%%%%%%%%%%
%% MISCELLANEOUS MACROS.
%%%%%%%%%%%%%%%%%%%%%%%%%%%%%%%%%%%%%%%%%%%%%%%%%%%%%%%%%%%%%%%%%

\DeclareMathOperator*{\argmax}{arg\,max}
\DeclareMathOperator*{\argmin}{arg\,min}
\renewcommand{\binom}[2]{\left(\genfrac{}{}{0pt}{}{#1}{#2}\right)}
\newcommand{\ceil}[1]{{\left\lceil{#1}\right\rceil}}
%\newcommand{\ffrac}[2]{{\nicefrac{#1}{#2}}}
%\newcommand{\fffrac}[2]{{\left.{#1}\middle/{#2}\right.}}
\newcommand{\floor}[1]{{\left\lfloor{#1}\right\rfloor}}
\DeclareMathOperator{\lcm}{lcm}
\newcommand{\ZZ}{{\mathbb{Z}}}
